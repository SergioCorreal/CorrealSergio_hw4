\documentclass[]{article}
\usepackage[margin=2cm]{geometry}
\usepackage{graphicx}

\begin{document}
\section{Ordinary Differential Equation ODE}

The next figures show the projectile trajectory for various angles. The maximum range is obtained when $\theta = 10^o$ and its value was 5.42252 m.

\begin{figure}[!h]{
	\begin{center}
	\includegraphics[scale=0.7]{45.png}
	\caption{Trajectory for $45^o$}
	\end{center}
}\end{figure}

\begin{figure}[!h]{
	\begin{center}
	\includegraphics[scale=0.7]{otherAngles.png}
	\caption{Trajectory for all the angles}
	\end{center}
}\end{figure}

\section{Partial Differential Equation PDE}
The images in this section were obtained solving the diffusion equation for various boundary conditions. Thermal diffusivity $\nu$ was taken as $1*10^{-4}$ $m^2/s$. Total time of simulation was 2500000 s with $dt = 100$ s and $dx=dy=1.0$ m. In general, the steady state is one in which temperature increases as a polynomial from one of the sides to the center of the square.

\begin{figure}[!htb]{
	\begin{center}
	\includegraphics[scale=0.7]{initial.png}
	\end{center}
}\end{figure}


	\begin{figure}[!htb]{
		\begin{center}
		\includegraphics[scale=0.7]{ci1.png}
		\end{center}
	}\end{figure}

	\begin{figure}[!htb]{
		\begin{center}
		\includegraphics[scale=0.7]{ci2.png}
		\end{center}
	}\end{figure}
	
	\begin{figure}[!htb]{
		\begin{center}
		\includegraphics[scale=0.7]{cf.png}
		\end{center}
	}\end{figure}
	

	\begin{figure}[!htb]{
		\begin{center}
		\includegraphics[scale=0.7]{oi1.png}
		\end{center}
	}\end{figure}

	\begin{figure}[!htb]{
		\begin{center}
		\includegraphics[scale=0.7]{oi2.png}
		\end{center}
	}\end{figure}
	
	\begin{figure}[!htb]{
		\begin{center}
		\includegraphics[scale=0.7]{of.png}
		\end{center}
	}\end{figure}

	\begin{figure}[!htb]{
		\begin{center}
		\includegraphics[scale=0.7]{pi1.png}
		\end{center}
	}\end{figure}

	\begin{figure}[!htb]{
		\begin{center}
		\includegraphics[scale=0.7]{pi2.png}
		\end{center}
	}\end{figure}
	
	\begin{figure}[!htb]{
		\begin{center}
		\includegraphics[scale=0.7]{pf.png}
		\end{center}
	}\end{figure}
	
	\begin{figure}[!htb]{
		\begin{center}
		\includegraphics[scale=0.7]{i1.png}
		\end{center}
	}\end{figure}

	\begin{figure}[!htb]{
		\begin{center}
		\includegraphics[scale=0.7]{i2.png}
		\end{center}
	}\end{figure}
	
	\begin{figure}[!htb]{
		\begin{center}
		\includegraphics[scale=0.7]{f.png}
		\end{center}
	}\end{figure}

	
\end{document}
